\documentclass[a4paper]{report}
\usepackage[italian]{babel}
\usepackage[utf8]{inputenc}
\usepackage{hyperref, graphicx, colortbl, gensymb}

\hypersetup{colorlinks=false, urlcolor=black, linkcolor=black}

\title{Ingegneria del Software \\ Software  per il monitoraggio di un reparto di terapia intensiva}
\author{Matteo Meneghetti\hspace{0.2cm}VR39987 \\  Stefano Cattonar\hspace{0.2cm}VR402549}

\date{Luglio 2019}

\begin{document}

\maketitle
\tableofcontents

\chapter*{Introduzione}
    L'esame del corso di Ingegneria del Software richiedeva lo sviluppo di un software applicativo da parte di gruppi di studenti, con la possibilità di scegliere tra tre possibili progetti.\\
    Noi abbiamo deciso di sviluppare il progetto riguardante il monitoraggio di un reparto di terapia intensiva di un ospedale.
    
\chapter{Requisiti}
    \section{Analisi del testo}
        
        Prendiamo il \href{run:TerapiaIntensiva.pdf}{testo} dell'elaborato da produrre ed analizziamolo per determinare i requisiti:
        \begin{quote}
            Si vuole progettare un sistema informatico per gestire l’acquisizione di dati e segnali in una divisione di terapia intensiva.  
 
            Per ogni paziente ricoverato sono registrati i dati anagrafici principali: codice sanitario univoco, cognome, nome, data e luogo di nascita. Al momento del ricovero in terapia intensiva è registrata la diagnosi di ingresso di ogni paziente.
        \end{quote}
        Il primo paragrafo delinea in modo generale il software che andrà prodotto, mentre il secondo da già dei requisiti importanti:\\
        \begin{itemize}
            \item I dati da salvare per ogni paziente ricoverato e quando registrarli.
            \item 
        \end{itemize}
        

\chapter{Sviluppo}
    \section{Tecnologie utilizzate}
        
    \section{Pattern utilizzati}


\chapter{Test}
    \section{Test Manuali}
    \section{Test Automatici}



    

\end{document}
